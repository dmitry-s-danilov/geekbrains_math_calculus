\section*{Задача 4}

Пусть элементарные высказывания означают:
\begin{itemize}
  \item \(A\) -- сегодня светит солнце
  \item \(B\) -- сегодня сыро
  \item \(C\) -- я поеду на дачу
\end{itemize}

Тогда сформулировать составные высказывания:
\begin{enumerate}
  \item \(\neg A \vee B \Rightarrow \neg C\)
  \item \(C \Rightarrow A \vee \neg B\)
\end{enumerate}

\subsection*{Решение}

\subsubsection*{1}

\[\neg A \vee B \Rightarrow \neg C\]

Если сегодня пасмурно (\(\neg A\)) или (\(\vee\)) сыро (\(B\)),
то (\(\Rightarrow\)) я не поеду на дачу (\(\neg C\)).

\subsubsection*{2.1}

\[C \Rightarrow A \vee \neg B\]

Я поеду на дачу (\(C\)),
значит (\(\Rightarrow\)) сегодня ясно (\(A\)) или (\(\vee\)) сухо (\(\neg B\)).

\subsubsection*{2.2. Примечание}

\[C \Leftarrow A \vee \neg B\]

Я поеду на дачу (\(C\)) при условии, что (\(\Leftarrow\))
сегодня ясно (\(A\)) или (\(\vee\)) сухо (\(\neg B\)).

\[C \Leftrightarrow A \vee \neg B\]

Я поеду на дачу (\(C\)) тогда, и только при условии, что (\(\Leftrightarrow\))
сегодня ясно (\(A\)) или (\(\vee\)) сухо (\(\neg B\)).

